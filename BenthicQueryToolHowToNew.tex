% Options for packages loaded elsewhere
\PassOptionsToPackage{unicode}{hyperref}
\PassOptionsToPackage{hyphens}{url}
%
\documentclass[
]{article}
\usepackage{lmodern}
\usepackage{amsmath}
\usepackage{ifxetex,ifluatex}
\ifnum 0\ifxetex 1\fi\ifluatex 1\fi=0 % if pdftex
  \usepackage[T1]{fontenc}
  \usepackage[utf8]{inputenc}
  \usepackage{textcomp} % provide euro and other symbols
  \usepackage{amssymb}
\else % if luatex or xetex
  \usepackage{unicode-math}
  \defaultfontfeatures{Scale=MatchLowercase}
  \defaultfontfeatures[\rmfamily]{Ligatures=TeX,Scale=1}
\fi
% Use upquote if available, for straight quotes in verbatim environments
\IfFileExists{upquote.sty}{\usepackage{upquote}}{}
\IfFileExists{microtype.sty}{% use microtype if available
  \usepackage[]{microtype}
  \UseMicrotypeSet[protrusion]{basicmath} % disable protrusion for tt fonts
}{}
\makeatletter
\@ifundefined{KOMAClassName}{% if non-KOMA class
  \IfFileExists{parskip.sty}{%
    \usepackage{parskip}
  }{% else
    \setlength{\parindent}{0pt}
    \setlength{\parskip}{6pt plus 2pt minus 1pt}}
}{% if KOMA class
  \KOMAoptions{parskip=half}}
\makeatother
\usepackage{xcolor}
\IfFileExists{xurl.sty}{\usepackage{xurl}}{} % add URL line breaks if available
\IfFileExists{bookmark.sty}{\usepackage{bookmark}}{\usepackage{hyperref}}
\hypersetup{
  pdftitle={Benthic Query Tool How To},
  hidelinks,
  pdfcreator={LaTeX via pandoc}}
\urlstyle{same} % disable monospaced font for URLs
\usepackage[margin=1in]{geometry}
\usepackage{graphicx}
\makeatletter
\def\maxwidth{\ifdim\Gin@nat@width>\linewidth\linewidth\else\Gin@nat@width\fi}
\def\maxheight{\ifdim\Gin@nat@height>\textheight\textheight\else\Gin@nat@height\fi}
\makeatother
% Scale images if necessary, so that they will not overflow the page
% margins by default, and it is still possible to overwrite the defaults
% using explicit options in \includegraphics[width, height, ...]{}
\setkeys{Gin}{width=\maxwidth,height=\maxheight,keepaspectratio}
% Set default figure placement to htbp
\makeatletter
\def\fps@figure{htbp}
\makeatother
\setlength{\emergencystretch}{3em} % prevent overfull lines
\providecommand{\tightlist}{%
  \setlength{\itemsep}{0pt}\setlength{\parskip}{0pt}}
\setcounter{secnumdepth}{-\maxdimen} % remove section numbering
\ifluatex
  \usepackage{selnolig}  % disable illegal ligatures
\fi

\title{Benthic Query Tool How To}
\author{}
\date{\vspace{-2.5em}}

\begin{document}
\maketitle

~~~~~~This page is designed to assist users with benthic and habitat
data acquisition and reporting. The application is divided into two
sections to meet analytical needs. The
\texttt{Single\ Station\ Query\ (Live\ CEDS\ Connection)} tab allows
single station queries directly from the production environment. The
\texttt{Multiple\ Station\ Query\ (Archived\ Data\ Refreshed\ Weekly)}
tab allows for more complicated spatial and temporal analyses of benthic
and habitat data.

~~~~~~All tables feature interactive column sorting, column visibility,
and data download features. Tables can be sorted by specific column
using the gray arrows next to the column of interest. The
\texttt{Column\ Visibility} button allows users to turn columns on or
off to aid data visualization. The \texttt{Copy} button copies table
data to the user's clipboard for use in other applications. Some tables
feature \texttt{Search} boxes where users may enter character strings to
interactively filter table contents.

\hypertarget{single-station-query-live-ceds-connection}{%
\subsubsection{Single Station Query (Live CEDS
Connection)}\label{single-station-query-live-ceds-connection}}

~~~~~~Queries from this tab are restricted to single stations, allowing
users access to the most recent data available. Nightly database
rebuilds ensure data entered the previous day are available the follow
morning.

~~~~~~Users begin the query process by typing a DEQ Station ID into the
text box on the \texttt{Station\ Data} tab. After clicking the
\texttt{Pull\ Station} button, the application compiles available
station data from the production CEDS environment and GIS REST services.
If an invalid DEQ Station ID is entered, a notification in the bottom
right corner will appear alerting users that the entered Station ID does
not exist. Upon retrieving data, the application plots the station on an
interactive map, identifies key station information in the
\texttt{Station\ Information} table, and compiles sampling history in
the \texttt{Sampling\ Summary} table. The interactive map has a Level
III Ecoregion and DEQ Assessment Region layers hidden that can be
plotted by hovering the mouse over the layers button in the top left
corner of the map. Links to the chosen station in CEDS and in the GIS
Staff app are available in the \texttt{Station\ Information} table.

~~~~~~After a station is retrieved, the user may use the
\texttt{Benthic\ Data} and \texttt{Habitat\ Data} tabs. The
\texttt{Benthic\ Data} tab allows users to analyze benthic results on
the fly using their SCI method of choice, filter results by specific
collection date ranges, view unrarified samples, and filter results by
replicate number. The \texttt{Sampling\ Metrics} tab averages SCI scores
by specific windows (reactive to the selected benthic data range),
details collector and taxonomist information, and highlights sample
information. The \texttt{Detailed\ SCI\ Results} tab allows users to
visualize SCI scores in an interactive plot and details SCI metrics by
BenSampID. The interactive SCI bar plot is color coded by season and
allows users to turn layers on and off in the legend on the right panel.
Additional features such as zooming, panning, and plot downloads are
available in the top right corner of the plot by hovering the mouse on
the plot. The \texttt{Raw\ Benthic\ Data} tab features subtabs that
organize raw benthic data by BenSampID in wide or long formats called
\texttt{Crosstab\ View} and \texttt{Long\ View}, respectively.

~~~~~~The \texttt{Benthic\ Data\ Visualization\ Tools\ Tab}, nested
beneath \texttt{Benthic\ Data} tab provides benthic analysis tools based
on user requests. Tools may be added to this area by contacting Emma
Jones
(\href{mailto:emma.jones@deq.virginia.gov}{\nolinkurl{emma.jones@deq.virginia.gov}})
with specific examples and use cases. At present, the tools available
include:

\begin{itemize}
\tightlist
\item
  Benthic Individuals BenSamp Crosstab- This tool summarizes Genus or
  Family level individuals by StationID, sample date, replicate number,
  and BenSampID. Tolerance values and individuals counted are updated
  based on the Genus/Family radio button selection directly above the
  table. Tool requested by Billy Van Wart.
\item
  BCG Attribute Information- This tool joins the taxa collected at the
  selected station and collection window with various BCG attribute
  levels from different regional BCG projects. Tool requested by Kelly
  Hazlegrove.
\end{itemize}

~~~~~~The \texttt{Habitat\ Data} tab populates the date range filter to
match the window chosen in the \texttt{Benthics\ Data} tab, but users
can click the \texttt{Filter\ Custom\ Date\ Range} button to choose a
different window. The \texttt{Sampling\ Metrics} tab averages Total
Habitat scores by specific windows (reactive to the selected habitat
data range), details field team and habitat observation metric
information, and highlights sample information. The
\texttt{Detailed\ Habitat\ Results} tab allows users to visualize Total
Habitat scores in an interactive plot and details Total Habitat scores
and habitat metrics by HabSampID. The interactive Total Habitat bar plot
is color coded by season and allows users to turn layers on and off in
the legend on the right panel. Additional features such as zooming,
panning, and plot downloads are available in the top right corner of the
plot by hovering the mouse on the plot. The \texttt{Raw\ Habitat\ Data}
tab features tabs that organize raw habitat values and observations by
HabSampID in wide or long formats called \texttt{Crosstab\ View} and
\texttt{Long\ View}, respectively.

\hypertarget{multiple-station-query-archived-data-refreshed-weekly}{%
\subsubsection{Multiple Station Query (Archived Data Refreshed
Weekly)}\label{multiple-station-query-archived-data-refreshed-weekly}}

~~~~~~Queries from this tab are flexible, pulling and analyzing multiple
stations by various spatial and temporal filters. The data available to
this portion of the application is refreshed weekly and does not access
a live version of the production CEDS environment. Users begin by
choosing how they wish to query data, either by spatial filters,
wildcard selection, or manually specifying stations.

~~~~~~With the \texttt{Spatial\ Filters} radio button selected, users
may enter query terms into three cross validated filter boxes (e.g.~if a
Basin is entered, then only VAHU6s inside that basin can be chosen or if
an Assessment Region is selected, only Basins and VAHU6s within that
Assessment Region can be selected). More than one selection can be
entered into any of these boxes. Additionally below the linebreak, two
additional filters may be applied on the results at the time of
querying. These filters first take all results from the stations that
apply to the interactive cross validated selection and then apply a
spatial intersection on the chosen Level 3 Ecoregion(s) and filters
stations to contain sites sampled within the date range. Any filters
left blank do not apply to the selection. After all the desired filters
are chosen, press the \texttt{Pull\ Stations} button to run the query.
Wildcard selection does not apply to the results from the
\texttt{Spatial\ Filters} query method.

~~~~~~With the \texttt{Wildcard\ Query} radio button selected, a
wildcard query can be entered in the text box to find StationID's like
the entered text. \textbf{Remember to use the \% character for any
wildcard requests}. If an invalid query is entered, a notification in
the bottom right corner will appear alerting users that the entered text
does not match any DEQ StationID's. Click the \texttt{Pull\ Stations}
button to query based on the wildcard text. Spatial filters do not apply
to the results of the \texttt{Wildcard\ Selection} query method.

~~~~~~With the \texttt{Manually\ Specify\ Stations} radio button
selected, specific stations can be entered in the text box to query
those StationIDs. The application will begin to filter available
stations based on user input text into the field. Multiple stations can
be entered into the single text field. Click the \texttt{Pull\ Stations}
button to query based on the stations identified in the text field.
Spatial filters do not apply to the results of the
\texttt{Manually\ Specify\ Stations} query method.

~~~~~~Upon retrieving data, the application plots the stations on an
interactive map, identifies key station information in the
\texttt{Station\ Information} table, and compiles sampling history in
the \texttt{Sampling\ Summary} table. The interactive map has a Level
III Ecoregion, DEQ Assessment Region, and VAHU6 layers hidden that can
be plotted by hovering the mouse over the layers button in the top left
corner of the map. Links to the chosen station in CEDS and in the GIS
Staff app are available in the \texttt{Station\ Information} table.

\hypertarget{multistation-selection-map-draw-feature}{%
\paragraph{Multistation Selection Map Draw
Feature}\label{multistation-selection-map-draw-feature}}

~~~~~~To further restrict stations, use the map drawing polygon or
rectangle buttons in the left side of the interactive map to draw
features around the stations of interest. The polygon draw tool allows
users to draw specific shapes to select just the stations desired. To
finish a shape, double click on the first vertex. To use the rectangle
draw tool, the user first selects the rectangle tool, then clicks on the
map holding the mouse until the shape contains the stations desired.
Releasing the mouse finishes the rectangle. Selected stations turn
yellow and the selections are reflected in the
\texttt{Station\ Information} and \texttt{Sampling\ Summary} tables. To
remove selections, click the trash can button on the left side of the
interactive map. The user can either clear all to remove all selections
at once, or the user can click individual rectangles and polygons to
remove.

~~~~~~After the stations are retrieved, the user may use the
\texttt{Benthic\ Data} and \texttt{Habitat\ Data} tabs. The
\texttt{Benthic\ Data} tab allows users to analyze benthic results on
the fly by filtering results on specific collection date ranges, view
unrarified samples, and filter results by replicate number. SCI methods
are assigned based on sample location. The \texttt{Sampling\ Metrics}
tab averages SCI scores across all stations by method, benthic date
range window, and season in the \texttt{SCI\ Averages\ by\ Selection}
table. Station SCI results are averaged by benthic date range window and
season in the \texttt{SCI\ Averages\ by\ Station} table. Collector and
taxonomist information are summarized across all selected stations. The
\texttt{Detailed\ SCI\ Results} tab details SCI scores and metrics by
StationID and BenSampID. The \texttt{Raw\ Benthic\ Data} tab features
tabs that organize raw benthic data by BenSampID in wide or long formats
called \texttt{Crosstab\ View} and \texttt{Long\ View}, respectively.

~~~~~~The \texttt{Benthic\ Data\ Visualization\ Tools\ Tab}, nested
beneath \texttt{Benthic\ Data} tab provides benthic analysis tools based
on user requests. Tools may be added to this area by contacting Emma
Jones
(\href{mailto:emma.jones@deq.virginia.gov}{\nolinkurl{emma.jones@deq.virginia.gov}})
with specific examples and use cases. At present, the tools available
include:

\begin{itemize}
\tightlist
\item
  SCI Seasonal Crosstab- This tool allows users to visualize Station
  metrics by replicate number, collection method and season, and SCI
  method. The metric chosen in the drop down box above the table
  controls which metric is spread across the sample season and year
  columns. Tool requested by Billy Van Wart.
\item
  Benthic Individuals BenSamp Crosstab- This tool summarizes Genus or
  Family level individuals by StationID, sample date, replicate number,
  and BenSampID. Tolerance values and individuals counted are updated
  based on the Genus/Family radio button selection directly above the
  table. Tool requested by Billy Van Wart.1
\item
  BCG Attribute Information- This tool joins the taxa collected at the
  selected station and collection window with various BCG attribute
  levels from different regional BCG projects. Tool requested by Kelly
  Hazlegrove.
\end{itemize}

~~~~~~The \texttt{Habitat\ Data} tab populates the date range filter to
match the window chosen in the \texttt{Benthics\ Data} tab. The
\texttt{Sampling\ Metrics} tab averages Total Habitat scores by the date
range window specified on the \texttt{Benthic\ Data} tab, season, and by
individual StationIDs across the selected date range and seasons. Field
team and habitat observation metrics information is summarized across
all selected sites and \texttt{Habitat\ Sample\ Information} table
details collection information for each HabSampID. The
\texttt{Detailed\ Habitat\ Results} tab lists Total Habitat scores and
habitat metrics by HabSampID. The \texttt{Raw\ Habitat\ Data} tab
features tabs that organize raw habitat values and observations by
HabSampID in wide or long formats called \texttt{Crosstab\ View} and
\texttt{Long\ View}, respectively.

\hypertarget{contact-information}{%
\subsubsection{Contact Information}\label{contact-information}}

Please contact Jason Hill
(\href{mailto:jason.hill@deq.virginia.gov}{\nolinkurl{jason.hill@deq.virginia.gov}})
and Emma Jones
(\href{mailto:emma.jones@deq.virginia.gov}{\nolinkurl{emma.jones@deq.virginia.gov}})
for questions regarding the application.

\end{document}
